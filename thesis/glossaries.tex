\newglossaryentry{dns}
{%
    name={DNS},
    description={Domain Name System. Ist für die Namensauflösung im
    Internet zuständig.},
    first={Domain Name System (DNS)},
    long={Domain Name System}
}

\newglossaryentry{http}
{%
    name={HTTP},
    description={Hypertext Transport Protocol. Das im Internet übliche
    Protokoll zur Übermittlung von Webseiten.},
    first={Hypertext Transport Protocol (HTTP)},
    long={Hypertext Transport Protocol}

}
\newglossaryentry{https}
{%
    name={HTTPS},
    description={Hypertext Transport Protocol Secure. Eine Erweiterung
    für HTTP, welche HTTP mit TLS versieht.},
    first={Hypertext Transport Protocol Secure (HTTPS)},
    long={Hypertext Transport Protocol Secure}

}
\newglossaryentry{cifs}
{%
    name={CIFS},
    description={Common Internet File System. Ein offenes
    Protokoll zum Filetransfer und diversen anderen Dienst.},
    first={Common Internet File System (CIFS)},
    long={Common Internet File System}

}
\newglossaryentry{voip}
{%
    name={VoIP},
    description={Voice over Internet Protocol. Die Übermittlung von
    Sprache über das Internet Protocol.},
    first={Voice Over Internet Protocol (VoIP)},
    long={Voice Over Internet Protocol}

}
\newglossaryentry{tuc}
{%
    name={TU Clausthal},
    description={Technische Universität Clausthal. 1775 gegründete
    Technische Universität im Oberharz (Niedersachsen).},
    first={Technische Universität Clausthal}
}
\newglossaryentry{efzn}
{%
    name={EFZN},
    description={Energie-Forschungszentrum Niedersachsen. Gemeinsames
    wissenschaftliches Zentrum der TU Clausthal, TU Braunschweig,
    Universität Göttingen, Universität Hannover und Universität
    Oldenburg.},
    first={Energie-Forschungszentrum Niedersachsen (EFZN)},
    long={Energie-Forschungszentrum Niedersachsen}
}
\newglossaryentry{rfc}
{%
    name={RFC},
    description={Requests for Comments (deutsch: Bitte um Kommentare).
    Eine Reihe von technischer Standards, welche sich mit dem Internet
    befassen.},
    first={Request For Comments (RFC)},
    long={Request For Comments},
    plural={RFCs},
    longplural={Requests for Comments}
}
\newglossaryentry{ip}
{%
    name={IP},
    description={Internet Protocol. Das Standardprotokoll auf der
    Vermittlungsschicht des OSI-Modells.},
    first={Internet Protocol (IP)},
    long={Internet Protocol}
}
\newglossaryentry{tcp}
{%
    name={TCP},
    description={Transport Control Protocol. Ein Protokoll der
    Transportschicht des OSI-Modells. Im Gegensatz zu UDP ist TCP auf
    Datenintegrität und eine verlässliche Verbindung ausgelegt.},
    first={Transport Control Protocol (TCP)},
    long={Transport Control Protocol}
}
\newglossaryentry{udp}
{%
    name={UDP},
    description={User Datagramm Protocol. Einfaches Protokoll auf der
    Transportschicht zum senden von Daten. UDP besitzt keine Mechanismen
    zur Sicherstellung von Datenintegrität.},
    first={User Datagram Protocol (UDP)},
    long={User Datagram Protocol}
}
\newglossaryentry{fqdn}
{%
    name={FQDN},
    description={Fully-Qualified Domain Name. Die Bezeichnung für einen
    vollwertigen DNS Hostnamen im Internet.},
    first={Fully-Qualified Domain Name (FQDN)},
    long={Fully-Qualified Domain Name},
    plural={FQDNs},
    longplural={Fully-Qualified Domain Names}
}
\newglossaryentry{nic}
{%
    name={NIC},
    description={Network Information Center. Verwaltung einer oder
    mehrerer Top-Level Domains im Internet.},
    first={Network Information Center (NIC)},
    long={Network Information Center}
}
\newglossaryentry{arpanet}
{%
    name={ARPANET},
    description={Advanced Research Projects Agency Network. Der
    Vorgänger des Internets und ehemaliges Projekt der US-Luftwaffe.},
    first={Advanced Research Projects Agency Network (ARPANET)},
    long={Advanced Research Projects Agency Network}
}
\newglossaryentry{ftp}
{%
    name={FTP},
    description={File Transfer Protocol. Datentransferprotokoll auf der
    Anwendungsschicht des OSI-Modells.},
    first={File Transfer Protocol (FTP)},
    long={File Transfer Protocol}
}
\newglossaryentry{tld}
{%
    name={TLD},
    description={Top-Level Domain. Höchste Stufe der DNS Auflösung im
    Internet.},
    first={Top-Level Domain (TLD)},
    long={Top-Level Domain},
    plural={TLDs},
    firstplural={Top-Level Domains (TLDs)}
}
\newglossaryentry{icann}
{%
    name={ICANN},
    description={Internet Corporation for Assigned Names and Numbers.
    Internationale Aufsichtsbehörde für das Internet. ICANN koordiniert
    die Vergabe von einmaligen IP-Adressen und DNS Hostnamen.},
    first={Internet Corporation for Assigned Names and Numbers (ICANN)},
    long={Internet Corporation for Assigned Names and Numbers}
}
\newglossaryentry{osi}
{%
    name={OSI-Modell},
    description={Open Systems Interconnection Model. Referenzmodell für
    Verbindungen im Internet.},
    first={Open Systems Interconnection Model (OSI-Modell)}
}
\newglossaryentry{restapi}
{%
    name={REST-API},
    description={Representational State Transfer Application Programming
    Interface. Auf HTTP basierende Schnittstelle zur Interaktion mit
    anderen Programmen.},
    first={Representational State Transfer Application Programming Interface (REST-API)}
}
\newglossaryentry{url}
{%
    name={URL},
    description={Uniform Resource Locator. Vollwertiger Bezeichner einer
    Internetadresse mit vorangestelltem Protokoll und nachgestellten
    Pfad zur angeforderten Datei.},
    first={Uniform Resource Locator (URL)}
}
\newglossaryentry{tls}
{%
    name={TLS},
    description={Transport Layer Security. Zusätzliche Schicht für
    diverse Protokolle für Verschlüsselung, Entschlüsselung und
    Authentifikation von Daten.},
    first={Transport Layer Security (TLS)}
}
\newglossaryentry{ssl}
{%
    name={SSL},
    description={Secure Sockets Layer. Die veraltete Bezeichnung für
    TLS.},
    first={Secure Sockets Layer (SSL)}
}
\newglossaryentry{rsa}
{%
    name={RSA},
    description={Asymmetrisches kryptographisches Verfahren.}
}
\newglossaryentry{pfs}
{%
    name={PFS},
    description={Perfect Forward Secrecy. Bezeichnung für ein Verfahren,
    welches sicherstellt, dass im Falle eines Schlüsselverlusts bereits
    gesendete Daten nicht mehr entschlüsselt werden können.},
    first={Perfect Forward Secrecy (PFS)}
}
\newglossaryentry{smb}
{%
    name={SMB},
    description={Server Message Block. Protokoll auf der
    Anwendungsschicht des OSI-Modells zur Übermittlung von Daten und
    anderen Diensten in Rechnernetzen.},
    first={Server Message Block}
}
\newglossaryentry{netbios}
{%
    name={NetBIOS},
    description={Network Basic Input Output System.
    Programmierschnittstelle zur Kommunikation zwischen zwei Programmen.},
    first={Network Basic Input Output System (NetBIOS)}
}
\newglossaryentry{iana}
{%
    name={IANA},
    description={Internet Assigned Numbers Authority. Abteilung der
    ICANN. Ist für die Vergabe von Nummern und Namen im Internet
    zuständig.},
    first={Internet Assigned Numbers Authority (IANA)}
}
\newglossaryentry{tsdb}
{%
    name={TDSB},
    description={Time Series Database. Eine per Zeit indexierte
    Datenbank. Optimiert auf große Mengen an Daten, die eine strikte
    Relation zu der Zeit besitzen.},
    first={Time Series Database (TSDB)},
    plural={TSDBs}
}
\newglossaryentry{sql}
{%
    name={SQL},
    description={Structured Query Language. Eine Sprache zur
    Beschreibung von relationalen Datenbanken und dessen Operationen auf
    eben diesen.},
    first={Structured Query Language (SQL)}
}
\newglossaryentry{wal}
{%
    name={WAL},
    description={Write-Ahead-Log. Ein Mechanismus in Prometheus zur
    Wiederherstellung von Metriken nach einem Crash.},
    first={Write-Ahead-Log (WAL)}
}
\newglossaryentry{blob}
{%
    name={BLOB},
    description={Binary Large Object. Besonders große Binärdateien.},
    first={Binary Large Object (BLOB)},
    plural={BLOBs}
}
\newglossaryentry{ram}
{%
    name={RAM},
    description={Read-Only Memory. Flüchtiger aber sehr schneller
    Speicher in Rechnersystemen.},
    first={Read-Only Memory (RAM)}
}
\newglossaryentry{ssh}
{%
    name={SSH},
    description={Secure Shell. Netzwerkprotokoll, welches
    Verschlüsselung bietet. Wird zum entfernten Aufruf einer
    Kommandozeile verwendet.},
    first={Secure Shell (SSH)}
}
\newglossaryentry{yaml}
{%
    name={YAML},
    description={Yet Another Markup Language. Vereinfachte
    Auszeichnungssprache für Datenstrukturen.},
    first={Yet Another Markup Language (YAML)}
}
