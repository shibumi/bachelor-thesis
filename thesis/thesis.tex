\documentclass[titlepage]{report}
\usepackage[ngerman]{babel}
\usepackage[backend=biber,style=numeric]{biblatex}
\addbibresource{literature.bib}
\usepackage{caption}
\usepackage{subcaption}
\usepackage{graphicx}
\usepackage[utf8]{inputenc}
\usepackage[T1]{fontenc}
\usepackage{url}
\usepackage{hyphenat}
\usepackage{glossaries}
\makeglossaries{}
\newglossaryentry{dns}
{%
    name={DNS},
    description={},
    first={Domain Name System (DNS)},
    long={Domain Name System}
}

\newglossaryentry{http}
{%
    name={HTTP},
    description={},
    first={Hypertext Transport Protocol (HTTP)},
    long={Hypertext Transport Protocol}

}
\newglossaryentry{https}
{%
    name={HTTPS},
    description={},
    first={Hypertext Transport Protocol Secure (HTTPS)},
    long={Hypertext Transport Protocol Secure}

}
\newglossaryentry{cifs}
{%
    name={CIFS},
    description={},
    first={Common Internet File System (CIFS)},
    long={Common Internet File System}

}
\newglossaryentry{voip}
{%
    name={VoIP},
    description={},
    first={Voice Over Internet Protocol (VoIP)},
    long={Voice Over Internet Protocol}

}
\newglossaryentry{tuc}
{%
    name={TU Clausthal},
    description={},
    first={Technische Universität Clausthal},
}
\newglossaryentry{efzn}
{%
    name={EFZN},
    description={},
    first={Energie-Forschungszentrum Niedersachsen (EFZN)},
    long={Energie-Forschungszentrum Niedersachsen}
}
\newglossaryentry{rfc}
{%
    name={RFC},
    description={},
    first={Request For Comments (RFC)},
    long={Request For Comments},
    plural={RFCs},
    longplural={Requests for Comments}
}
\newglossaryentry{ip}
{%
    name={IP},
    description={},
    first={Internet Protocol (IP)},
    long={Internet Protocol}
}
\newglossaryentry{tcp}
{%
    name={TCP},
    description={},
    first={Transport Control Protocol (TCP)},
    long={Transport Control Protocol}
}
\newglossaryentry{udp}
{%
    name={UDP},
    description={},
    first={User Datagram Protocol (UDP)},
    long={User Datagram Protocol}
}
\newglossaryentry{fqdn}
{%
    name={FQDN},
    description={},
    first={Fully-Qualified Domain Name (FQDN)},
    long={Fully-Qualified Domain Name},
    plural={FQDNs},
    longplural={Fully-Qualified Domain Names}
}
\newglossaryentry{nic}
{%
    name={NIC},
    description={},
    first={Network Information Center (NIC)},
    long={Network Information Center}
}
\newglossaryentry{arpanet}
{%
    name={ARPANET},
    description={},
    first={Advanced Research Projects Agency Network (ARPANET)},
    long={Advanced Research Projects Agency Network}
}
\newglossaryentry{ftp}
{%
    name={FTP},
    description={},
    first={File Transfer Protocol (FTP)},
    long={File Transfer Protocol}
}
\newglossaryentry{tld}
{%
    name={TLD},
    description={},
    first={Top-Level Domain (TLD)},
    long={Top-Level Domain},
    plural={TLDs},
    firstplural={Top-Level Domains (TLDs)}
}
\newglossaryentry{icann}
{%
    name={ICANN},
    description={},
    first={Internet Corporation for Assigned Names and Numbers (ICANN)},
    long={Internet Corporation for Assigned Names and Numbers}
}
\newglossaryentry{osi}
{%
    name={OSI-Modell},
    description={},
    first={Open Systems Interconnection Model (OSI-Modell)}
}
\newglossaryentry{restapi}
{%
    name={REST-API},
    description={},
    first={Representational State Transfer Application Programming Interface (REST-API)}
}
\newglossaryentry{url}
{%
    name={URL},
    description={},
    first={Uniform Resource Locator (URL)}
}
\newglossaryentry{tls}
{%
    name={TLS},
    description={},
    first={Transport Layer Security (TLS)}
}
\newglossaryentry{ssl}
{%
    name={SSL},
    description={},
    first={Secure Sockets Layer (SSL)}
}
\newglossaryentry{rsa}
{%
    name={RSA},
    description={}
}
\newglossaryentry{pfs}
{%
    name={PFS},
    description={},
    first={Perfect Forward Secrecy (PFS)}
}
\newglossaryentry{smb}
{%
    name={SMB},
    description={},
    first={Server Message Block}
}
\newglossaryentry{netbios}
{%
    name={NetBIOS},
    description={},
    first={Network Basic Input Output System}
}
\newglossaryentry{iana}
{%
    name={IANA},
    description={},
    first={Internet Assigned Numbers Authority (IANA)}
}

\title{Evaluierung eines Messpunkte-Clusters für Netzwerktests auf dem
Campus der TU Clausthal}
\author{Christian, Rebischke\\
\gls{tuc}\\
Rechenzentrum\\
Email: Christian.Rebischke@tu-clausthal.de}
\begin{document}
\maketitle
\chapter*{Danksagung}
Ich bedanke mich bei dem Rechenzentrum der \gls{tuc}, insbesondere bei
Dipl.\hyp{}Math. Christian Strauf. Das Thema dieser Bachelorarbeit
beruht auf seiner Idee und war mein Ansporn mich mit diesem Thema näher
auseinanderzusetzen. Desweiteren danke ich Herrn Prof. Dr.\hyp{}Ing. Dr.
rer. nat. habil. Harald Richter für die Unterstützung aus akademischer
Seite.
\tableofcontents
\chapter*{Vorwort}
\addcontentsline{toc}{chapter}{Vorwort}
Es gibt nun seit mehr als 20 Jahren das Internet und keine Technologie
ist wohl über so kurze Zeit so alltäglich geworden. Das Internet hat es
geschafft Einzug zu erhalten in Arbeit, Privatleben und auch Forschung
und Lehre. Nahezu in allen wissenschaftlichen Diszplinen spielt das
Internet und die damit verknüpfte Informationstechnologie eine Rolle.
Sei es die Industrie 4.0 mit ihren Cyber-physischen Systemen, dem
schnellen Abgleich von DNA-Informationen über das Netz in der
angewandten Biologie, dem Sammeln von Krankheitsdaten in der Medizin
oder das Verarbeiten von Datenmengen gigantischen Ausmaßes im
Finanzsektor. All diese Beispiele sind nur möglich durch immer größere
Technologiesprünge in der Informatik und dem immer weiteren Ausbau des
Internets. Da ist es nicht verwunderlich, dass der UN-Menschenrechtsrat das
Internet zu einem Menschenrecht\cite{UNHRC} erklärt hat und umso weniger
verwunderlich ist es, dass die Vernetzung von Computersystemen auch auf
dem Campus der \gls{tuc} eine Rolle spielt,
nicht nur für Forschung und Lehre, sondern natürlich auch für den
täglichen Betrieb. Eine Schlüsselposition nimmt dabei das Rechenzentrum
der \gls{tuc} ein. Das Rechenzentrum bildet die
Basis für die Vernetzung der einzelnen Fakultäten untereinander, die
Vernetzung zwischen Fakultäten und Firmen aus der freien Wirtschaft,
sowie auch die Vernetzung zwischen der \gls{tuc} 
und anderen Universitäten weltweit. Dementsprechend wichtig ist ein
stabiles Netz für den täglichen Betrieb. In dieser Bachelorarbeit widme
ich mich deshalb der technischen Umsetzung eines verteilten
Monitoring-Systems zur Überwachung der Netzwerkqualität zwischen
einzelnen Endpunkten und Kernsystemen die für einen problemlosen
Netzbetrieb nötig sind. Das Rechenzentrum der \gls{tuc} dient bei dieser
Bachelorarbeit als Auftraggeber.
\chapter*{Problemstellung}
\addcontentsline{toc}{chapter}{Problemstellung}
Das Netz der \gls{tuc} erstreckt sich über mehrere
Standorte. Teilweise liegen diese Standorte nicht in Clausthal
selbst, wie beispielsweise das \gls{efzn} in Goslar. Dementsprechend
schwierig gestaltet sich die Wartung und der Betrieb
des Netzes. So kann auf Netzeinbrüche etwa nur reaktiv
nach Meldung des Problems reagiert werden. Es existiert
zwar ein Monitoring-System, welches die Verfügbarkeit von
einzelnen Diensten überprüft, jedoch erfolgt diese Messung
nur von einem Punkt aus und gibt nur binäre Statuswerte
zurück (Dienst läuft oder Dienst läuft nicht). Dementsprechend
fehlen Informationen um die Verfügbarkeit von Diensten und
deren vollständige Funktion von mehreren Messpunkten aus
zu garantieren. Beispielsweise ist es möglich, dass ein Dienst
zwar vom zentralen Monitoring-Server aus erreichbar ist, aber
aus einem einzelnen Institut der Zugriff auf den Dienst nur
eingeschränkt oder sogar gar nicht möglich ist. Das Rechenzentrum der
\gls{tuc} bietet mehrere Kerndienste an. Dazu
gehören:
\begin{itemize}
    \item \gls{dns}
    \item Diverse Webdienste basierend auf:
    \begin{itemize}
        \item \gls{http}
        \item \gls{https}
    \end{itemize}
    \item \gls{cifs}
    \item \gls{voip}
\end{itemize}
Besonders Dienste wie \gls{dns} sind auf schnelle
Verbindungen angewiesen. Eine zu hohe Latenz zwischen einem Client und
dem Dienst führt unweigerlich zu für den Nutzer sichtbaren Konsequenzen
(zum Beispiel verzögerte Seitenaufrufe beim Web-Browsing). Noch mehr ins
Gewicht fallen Latenzen bei \gls{voip}, dort sind Latenzen
oder ein Jitter (die Varianz der Laufzeit der Datenpakete\cite{JITTER})
leicht auszumachen. Nämlich an abgehakten Telefonaten oder Rauschen in
der Leitung. Was also fehlt ist ein Netz aus verteilten Messpunkten,
dass es ermöglicht die Erreichbarkeit einzelner Dienste periodisch und
über einen längeren Zeitraum zu beobachten. Dies hätte zwei Vorteile.
Zum einen lässt sich so der Zustand des Campus-Netzwerks besser
erfassen, da Tests nicht nur von einem zentralen Monitoring-System aus
gestartet werden. Zum anderen können die gewonnenen Daten weiter
verwertet, grafisch aufbereitet und zum Beispiel für die Erstellung von
Langzeitstatistiken über die Gesundheit des Netzwerks genutzt werden.
Weiterhin könnten im Fall eines Ausfalls natürlich die zuständigen
Netzadministratoren benachrichtigt werden, im Idealfall durch gewohnte
Kommunikationswege wie Email. Mit dem Wandel von einem zentralen zu
einem dezentralen Monitoring-System entsteht allerdings auch mehr
Arbeitsaufwand. Denn auch diese Systeme müssen gewartet werden. Im
nachfolgenden Kapitel werden aus dieser Problemstellung die nötigen
Anforderungen abgeleitet und ein erster Lösungsansatz für das Problem
erstellt.
\chapter*{Herleitung eines Lösungsansatzes}
\addcontentsline{toc}{chapter}{Herleitung eines Lösungsansatzes}
\section*{Anforderungsanalyse}
\addcontentsline{toc}{section}{Anforderungsanalyse}
Nachfolgend werden die ermittelten funktionalen und nichtfunktionalen
Anforderungen erläutert.
\chapter*{Fazit}
\addcontentsline{toc}{chapter}{Fazit}
\nocite{*}
\printbibliography{}
\listoffigures
\printglossary{}
\end{document}
