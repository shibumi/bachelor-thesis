\documentclass[titlepage]{report}
\usepackage[ngerman]{babel}
\usepackage[backend=biber,style=numeric]{biblatex}
\addbibresource{literature.bib}
\usepackage{caption}
\usepackage{subcaption}
\usepackage{graphicx}
\usepackage[utf8]{inputenc}
\usepackage[T1]{fontenc}
\usepackage{url}
\usepackage{hyphenat}
\usepackage{glossaries}
\usepackage{array}
\usepackage{calc}
\usepackage{booktabs}
\makeglossaries{}
\newglossaryentry{dns}
{%
    name={DNS},
    description={},
    first={Domain Name System (DNS)},
    long={Domain Name System}
}

\newglossaryentry{http}
{%
    name={HTTP},
    description={},
    first={Hypertext Transport Protocol (HTTP)},
    long={Hypertext Transport Protocol}

}
\newglossaryentry{https}
{%
    name={HTTPS},
    description={},
    first={Hypertext Transport Protocol Secure (HTTPS)},
    long={Hypertext Transport Protocol Secure}

}
\newglossaryentry{cifs}
{%
    name={CIFS},
    description={},
    first={Common Internet File System (CIFS)},
    long={Common Internet File System}

}
\newglossaryentry{voip}
{%
    name={VoIP},
    description={},
    first={Voice Over Internet Protocol (VoIP)},
    long={Voice Over Internet Protocol}

}
\newglossaryentry{tuc}
{%
    name={TU Clausthal},
    description={},
    first={Technische Universität Clausthal},
}
\newglossaryentry{efzn}
{%
    name={EFZN},
    description={},
    first={Energie-Forschungszentrum Niedersachsen (EFZN)},
    long={Energie-Forschungszentrum Niedersachsen}
}
\newglossaryentry{rfc}
{%
    name={RFC},
    description={},
    first={Request For Comments (RFC)},
    long={Request For Comments}
}

\title{Evaluierung eines Messpunkte-Clusters für Netzwerktests auf dem
Campus der TU Clausthal}
\author{Christian, Rebischke\\
\gls{tuc}\\
Rechenzentrum\\
Email: Christian.Rebischke@tu-clausthal.de}
\begin{document}
\maketitle
\chapter*{Danksagung}
Ich bedanke mich bei dem Rechenzentrum der \gls{tuc}, insbesondere bei
Dipl.\hyp{}Math. Christian Strauf. Das Thema dieser Bachelorarbeit
beruht auf seiner Idee und war mein Ansporn mich mit diesem Thema näher
auseinanderzusetzen. Desweiteren danke ich Herrn Prof. Dr.\hyp{}Ing. Dr.
rer. nat. habil. Harald Richter für die Unterstützung aus akademischer
Seite.
\tableofcontents
\chapter*{Vorwort}
\addcontentsline{toc}{chapter}{Vorwort}
Es gibt nun seit mehr als 20 Jahren das Internet und keine Technologie
ist wohl über so kurze Zeit so alltäglich geworden. Das Internet hat es
geschafft Einzug zu erhalten in Arbeit, Privatleben und auch Forschung
und Lehre. Nahezu in allen wissenschaftlichen Diszplinen spielt das
Internet und die damit verknüpfte Informationstechnologie eine Rolle.
Sei es die Industrie 4.0 mit ihren Cyber-physischen Systemen, dem
schnellen Abgleich von DNA-Informationen über das Netz in der
angewandten Biologie, dem Sammeln von Krankheitsdaten in der Medizin
oder das Verarbeiten von Datenmengen gigantischen Ausmaßes im
Finanzsektor. All diese Beispiele sind nur möglich durch immer größere
Technologiesprünge in der Informatik und dem immer weiteren Ausbau des
Internets. Da ist es nicht verwunderlich, dass der UN-Menschenrechtsrat das
Internet zu einem Menschenrecht\cite{UNHRC} erklärt hat und umso weniger
verwunderlich ist es, dass die Vernetzung von Computersystemen auch auf
dem Campus der \gls{tuc} eine Rolle spielt,
nicht nur für Forschung und Lehre, sondern natürlich auch für den
täglichen Betrieb. Eine Schlüsselposition nimmt dabei das Rechenzentrum
der \gls{tuc} ein. Das Rechenzentrum bildet die
Basis für die Vernetzung der einzelnen Fakultäten untereinander, die
Vernetzung zwischen Fakultäten und Firmen aus der freien Wirtschaft,
sowie auch die Vernetzung zwischen der \gls{tuc}
und anderen Universitäten weltweit. Dementsprechend wichtig ist ein
stabiles Netz für den täglichen Betrieb. In dieser Bachelorarbeit widme
ich mich deshalb der technischen Umsetzung eines verteilten
Monitoring-Systems zur Überwachung der Netzwerkqualität zwischen
einzelnen Endpunkten und Kernsystemen die für einen problemlosen
Netzbetrieb nötig sind. Das Rechenzentrum der \gls{tuc} dient bei dieser
Bachelorarbeit als Auftraggeber.
\chapter*{Problemstellung}
\addcontentsline{toc}{chapter}{Problemstellung}
Das Netz der \gls{tuc} erstreckt sich über mehrere
Standorte. Teilweise liegen diese Standorte nicht in Clausthal
selbst, wie beispielsweise das \gls{efzn} in Goslar. Dementsprechend
schwierig gestaltet sich die Wartung und der Betrieb
des Netzes. So kann auf Netzeinbrüche etwa nur reaktiv
nach Meldung des Problems reagiert werden. Es existiert
zwar ein Monitoring-System, welches die Verfügbarkeit von
einzelnen Diensten überprüft, jedoch erfolgt diese Messung
nur von einem Punkt aus und gibt nur binäre Statuswerte
zurück (Dienst läuft oder Dienst läuft nicht). Dementsprechend
fehlen Informationen um die Verfügbarkeit von Diensten und
deren vollständige Funktion von mehreren Messpunkten aus
zu garantieren. Beispielsweise ist es möglich, dass ein Dienst
zwar vom zentralen Monitoring-Server aus erreichbar ist, aber
aus einem einzelnen Institut der Zugriff auf den Dienst nur
eingeschränkt oder sogar gar nicht möglich ist. Das Rechenzentrum der
\gls{tuc} bietet mehrere Kerndienste an. Dazu
gehören:
\begin{itemize}
    \item \gls{dns}
    \item Diverse Webdienste basierend auf:
    \begin{itemize}
        \item \gls{http}
        \item \gls{https}
    \end{itemize}
    \item \gls{cifs}
    \item \gls{voip}
\end{itemize}
Besonders Dienste wie \gls{dns} sind auf schnelle
Verbindungen angewiesen. Eine zu hohe Latenz zwischen einem Client und
dem Dienst führt unweigerlich zu für den Nutzer sichtbaren Konsequenzen
(zum Beispiel verzögerte Seitenaufrufe beim Web-Browsing). Noch mehr ins
Gewicht fallen Latenzen bei \gls{voip}, dort sind Latenzen
oder ein Jitter (die Varianz der Laufzeit der Datenpakete\cite{JITTER})
leicht auszumachen. Nämlich an abgehakten Telefonaten oder Rauschen in
der Leitung. Was also fehlt ist ein Netz aus verteilten Messpunkten,
dass es ermöglicht die Erreichbarkeit einzelner Dienste periodisch und
über einen längeren Zeitraum zu beobachten. Dies hätte zwei Vorteile.
Zum einen lässt sich so der Zustand des Campus-Netzwerks besser
erfassen, da Tests nicht nur von einem zentralen Monitoring-System aus
gestartet werden. Zum anderen können die gewonnenen Daten weiter
verwertet, grafisch aufbereitet und zum Beispiel für die Erstellung von
Langzeitstatistiken über die Gesundheit des Netzwerks genutzt werden.
Weiterhin könnten im Fall eines Ausfalls natürlich die zuständigen
Netzadministratoren benachrichtigt werden, im Idealfall durch gewohnte
Kommunikationswege wie Email. Mit dem Wandel von einem zentralen zu
einem dezentralen Monitoring-System entsteht allerdings auch mehr
Arbeitsaufwand. Denn auch diese Systeme müssen gewartet werden. Im
nachfolgenden Kapitel werden aus dieser Problemstellung die nötigen
Anforderungen abgeleitet und ein erster Lösungsansatz für das Problem
erstellt.
\chapter*{Herleitung eines Lösungsansatzes}
\addcontentsline{toc}{chapter}{Herleitung eines Lösungsansatzes}
\section*{Technische Grundlagen}
\addcontentsline{toc}{section}{Technische Grundlagen}
\section*{Anforderungsanalyse}
\addcontentsline{toc}{section}{Anforderungsanalyse}
Nachfolgend werden die ermittelten funktionalen und nichtfunktionalen
Anforderungen erläutert. Funktionale Anforderungen stellen das
``eigentliche Systemverhalten und die jeweiligen Funktionen des zu
erstellenden Produkts''\cite[S. 20]{BPSE} dar, also die grundlegenden
Aufgaben der Software im Bezug auf die Problemstellung. Die
nichtfunktionalen Anforderungen dagegen sind besonders. Sie umfassen
Anforderungen wie Sicherheit, nachträgliche Erweiterbarkeit,
Testbarkeit, also Anforderungen die erst nach der Entwicklung
mess\hyp{} oder testbar werden\cite[S. 292]{SNFA}. Um die funktionalen
und nichtfunktionalen Anforderungen besser einordnen zu können, werden
folgende Schlüsselwörter zum Kennzeichnen für Anforderungen nach
\gls{rfc} 2119\cite{RFC2119} definiert:
\begin{description}
    \item[ERFORDERLICH] ist eine absolute Anforderung an die Software. Alle
        Anforderungen die mit \textbf{ERFORDERLICH} markiert sind,
        \textbf{MÜSSEN} implementiert worden sein sein.
    \item[VERBOTEN] beschreibt eine negative Anforderung. Demnach eine
        Anforderung die auf keinen Fall implementiert werden darf.
    \item[EMPFOHLEN] ist eine Anforderung die implementiert werden
        \textbf{SOLLTE} aber nicht \textbf{MUSS}. Dies ist der Fall bei
        Anforderungen die aus nachvollziehbaren Gründen nicht
        implementiert worden sind.
    \item[NICHT EMPFOHLEN] ist das Gegenteil von \textbf{EMPFOHLEN}. Es
        handelt sich hier um Anforderungen die vermieden werden sollten.
    \item[OPTIONAL] ist eine Anforderung die implementiert werden
        \textbf{KANN}. Diese Art von Anforderungen sind
        zusätzliches Extra und nicht nötig für die Grundfunktion der
        Software.
\end{description}
\textbf{Anmerkung}: Das \gls{rfc} 2119 ist im Original in Englisch. Ich
habe mich zur übersetzung der Schlüsselwörter auf die Übersetzung der
Schweizer Firma Adfinis SyGroup AG gestützt\cite{RFC2119DE}.
\section*{Funktionale Anforderungen}
\begin{center}
\begin{tabular}{p{0.7\textwidth-\tabcolsep}>{\raggedleft\arraybackslash}p{0.3\textwidth-\tabcolsep}}\toprule
    \textbf{FA1: Zeitmessung von \gls{dns}-Abfragen } & \textbf{Priorität: MUSS} \\\midrule
	\multicolumn{2}{p{\textwidth-\tabcolsep}}{%
    Die Software \textbf{MUSS} die Zeit messen können, die vergeht
    zwischen einer \gls{dns}-Abfrage und der Antwort von einem
    \gls{dns}-Server}\\\bottomrule
\end{tabular}
\end{center}
\begin{center}
\begin{tabular}{p{0.7\textwidth-\tabcolsep}>{\raggedleft\arraybackslash}p{0.3\textwidth-\tabcolsep}}\toprule
    \textbf{FA2: Zeitmessung von \gls{http}-Abfragen } & \textbf{Priorität: MUSS} \\\midrule
	\multicolumn{2}{p{\textwidth-\tabcolsep}}{%
    Die Software \textbf{MUSS} die Zeit messen können, die vergeht
    zwischen einer \gls{http}-Abfrage und der Antwort von einem
    \gls{http}-Server}\\\bottomrule
\end{tabular}
\end{center}
\begin{center}
\begin{tabular}{p{0.7\textwidth-\tabcolsep}>{\raggedleft\arraybackslash}p{0.3\textwidth-\tabcolsep}}\toprule
    \textbf{FA3: Zeitmessung von \gls{https}-Abfragen } & \textbf{Priorität: MUSS} \\\midrule
	\multicolumn{2}{p{\textwidth-\tabcolsep}}{%
    Die Software \textbf{MUSS} die Zeit messen können, die vergeht
    zwischen einer \gls{https}-Abfrage und der Antwort von einem
    \gls{https}-Server}\\\bottomrule
\end{tabular}
\end{center}
\begin{center}
\begin{tabular}{p{0.7\textwidth-\tabcolsep}>{\raggedleft\arraybackslash}p{0.3\textwidth-\tabcolsep}}\toprule
    \textbf{FA4: Zeitmessung von \gls{cifs}-Abfragen } & \textbf{Priorität: MUSS} \\\midrule
	\multicolumn{2}{p{\textwidth-\tabcolsep}}{%
    Die Software \textbf{MUSS} die Zeit messen können, die vergeht
    zwischen einer \gls{cifs}-Abfrage und der Antwort von einem
    \gls{cifs}-Server}\\\bottomrule
\end{tabular}
\end{center}
\begin{center}
\begin{tabular}{p{0.7\textwidth-\tabcolsep}>{\raggedleft\arraybackslash}p{0.3\textwidth-\tabcolsep}}\toprule
    \textbf{FA5: Zeitmessung von \gls{voip}-Abfragen } & \textbf{Priorität: MUSS} \\\midrule
	\multicolumn{2}{p{\textwidth-\tabcolsep}}{%
    Die Software \textbf{MUSS} die Zeit messen können, die vergeht
    zwischen einer \gls{voip}-Abfrage und der Antwort von einem
    \gls{voip}-Server}\\\bottomrule
\end{tabular}
\end{center}
\begin{center}
\begin{tabular}{p{0.7\textwidth-\tabcolsep}>{\raggedleft\arraybackslash}p{0.3\textwidth-\tabcolsep}}\toprule
    \textbf{FA6: Grafische Aufbereitung } & \textbf{Priorität: MUSS} \\\midrule
	\multicolumn{2}{p{\textwidth-\tabcolsep}}{%
    Die Software \textbf{MUSS} die Zeit messen können, die vergeht
    zwischen einer \gls{voip}-Abfrage und der Antwort von einem
    \gls{voip}-Server}\\\bottomrule
\end{tabular}
\end{center}
\chapter*{Fazit}
\addcontentsline{toc}{chapter}{Fazit}
\nocite{*}
\printbibliography{}
\listoffigures
\printglossary{}
\end{document}
