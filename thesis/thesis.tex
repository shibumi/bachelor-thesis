\documentclass[titlepage]{report}
\usepackage[ngerman]{babel}
\usepackage[backend=biber,style=numeric]{biblatex}
\addbibresource{literature.bib}
\usepackage{caption}
\usepackage{subcaption}
\usepackage{graphicx}
\usepackage[utf8]{inputenc}
\usepackage[T1]{fontenc}
\usepackage{url}
\usepackage{hyphenat}
\title{Evaluierung eines Messpunkte-Clusters für Netzwerktests auf dem
Campus der TU Clausthal}
\author{Christian, Rebischke\\
Technische Universität Clausthal\\
Rechenzentrum\\
Email: Christian.Rebischke@tu-clausthal.de}
\begin{document}
\maketitle
\chapter*{Danksagung}
Ich bedanke mich bei dem Rechenzentrum der Technischen Universität
Clausthal, insbesondere bei Dipl.\hyp{}Math. Christian Strauf. Das Thema
dieser Bachelorarbeit beruht auf seiner Idee und war mein Ansporn mich
mit diesem Thema näher auseinanderzusetzen. Desweiteren danke ich Herrn
Prof. Dr.\hyp{}Ing. Dr. rer. nat. habil. Harald Richter für die Unterstützung
aus akademischer Seite.
\tableofcontents
\chapter*{Einleitung}
\addcontentsline{toc}{chapter}{Einleitung}
\section*{Vorwort}
\addcontentsline{toc}{section}{Vorwort}
Es gibt nun seit mehr als 20 Jahren das Internet und keine Technologie
ist wohl über so kurze Zeit so alltäglich geworden. Das Internet hat es
geschafft Einzug zu erhalten in Arbeit, Privatleben und auch Forschung
und Lehre. Nahezu in allen wissenschaftlichen Diszplinen spielt das
Internet und die damit verknüpfte Informationstechnologie eine Rolle.
Sei es die Industrie 4.0 mit ihren Cyber-physischen Systemen, dem
schnellen Abgleich von DNA-Informationen über das Netz in der
angewandten Biologie, dem Sammeln von Krankheitsdaten in der Medizin
oder das Verarbeiten von Datenmengen gigantischen Ausmaßes im
Finanzsektor. All diese Beispiele sind nur möglich durch immer größere
Technologiesprünge in der Informatik und dem immer weiteren Ausbau des
Internets. Da ist es nicht verwunderlich, dass der UN-Menschenrechtsrat das
Internet zu einem Menschenrecht\cite{UNHRC} erklärt hat und umso weniger
verwunderlich ist es, dass die Vernetzung von Computersystemen auch auf
dem Campus der Technischen Universität Clausthal eine Rolle spielt,
nicht nur für Forschung und Lehre, sondern natürlich auch für den
täglichen Betrieb. Eine Schlüsselposition nimmt dabei das Rechenzentrum
der Technischen Universität Clausthal ein. Das Rechenzentrum bildet die
Basis für die Vernetzung der einzelnen Fakultäten untereinander, die
Vernetzung zwischen Fakultäten und Firmen aus der freien Wirtschaft,
sowie auch die Vernetzung zwischen der Technischen Universität Clausthal
und anderen Universitäten weltweit. Dementsprechend wichtig ist ein
stabiles Netz für den täglichen Betrieb. In dieser Bachelorarbeit widme
ich mich deshalb der technischen Umsetzung eines verteilten
Monitoring-Systems zur Überwachung der Netzwerkqualität zwischen
einzelnen Endpunkten und Kernsystemen die für einen problemlosen
Netzbetrieb nötig sind. Das Rechenzentrum der Technischen Universität
Clausthal dient bei dieser Bachelorarbeit als Auftraggeber.
\section*{Problemstellung}
\addcontentsline{toc}{section}{Problemstellung}
Das Netz der TU Clausthal erstreckt sich über mehrere
Standorte. Teilweise liegen diese Standorte nicht in Clausthal
selbst, wie beispielsweise das EFZN in Goslar. Dementsprechend
schwierig gestaltet sich die Wartung und der Betrieb
des Netzes. So kann auf Netzeinbrüche etwa nur reaktiv
nach Meldung des Problems reagiert werden. Es existiert
zwar ein Monitoring-System, welches die Verfügbarkeit von
einzelnen Diensten überprüft, jedoch erfolgt diese Messung
nur von einem Punkt aus und gibt nur binäre Statuswerte
zurück (Dienst läuft oder Dienst läuft nicht). Dementsprechend
fehlen Informationen um die Verfügbarkeit von Diensten und
deren vollständige Funktion von mehreren Messpunkten aus
zu garantieren. Beispielsweise ist es möglich, dass ein Dienst
zwar vom zentralen Monitoring-Server aus erreichbar ist, aber
aus einem einzelnen Institut der Zugriff auf den Dienst nur
eingeschränkt oder sogar gar nicht möglich ist. Besonders
Dienste wie Domain Name System (DNS) sind auf schnelle
Verbindungen angewiesen. Eine zu hohe Latenz zwischen
einem Client und dem Dienst führt unweigerlich zu für den
Nutzer sichtbaren Konsequenzen (Zum Beispiel verzögerte
Seitenaufrufe beim Web-Browsing). Aber auch andere Dienste sind
essentiell für einen problemlosen Netzbetrieb.
\chapter*{Definitionen}
\addcontentsline{toc}{chapter}{Definitionen}
\chapter*{Fazit}
\addcontentsline{toc}{chapter}{Fazit}
\nocite{*}
\printbibliography{}
\listoffigures
\end{document}
